\chapter{CONCLUSION}

In this project, we developed a basic social media server that implements parts of the \cite{ActivityPub} protocol. This allows users on our platform to interact with others on existing servers that follow the same protocol. In addition to our server, we also designed a clean and clutter-free frontend that supports not only our server but also other similar platforms, such as Mastodon\cite{mastodon}.

Users of our frontend can write mathematical text within plain text posts by leveraging the power of LaTeX through MathJax\cite{MathJax}.

Our server currently supports a single user, but with the ability to create multiple accounts for different purposes. Despite this single-user limitation, it is still possible to form a broader social network through ActivityPub, enabling communication across federated services. As a result, we have created a social media platform tailored for scientific and mathematical discussions while maintaining interoperability with the existing user base of ActivityPub-compliant services.

\section{Further Improvments}
We have implemented a basic social media platform with minimal interations. As such, there are few improvments to be made:
\begin{itemize}
  \item The server can be extended to support multiple users.
  \item The frontend can be enhanced to allow editing of posted content and provide more granular privacy settings.
  \item The server can be improved with robust moderation tools, such as blocking and filtering.
  \item LaTeX/MathJax auto-completion can be added to enhance mathematical input in the frontend.
\end{itemize}
