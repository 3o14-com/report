\chapter{LITERATURE REVIEW}

\section{Related Works}

The Fediverse connects various decentralized social networks, allowing users to interact across different platforms. This section reviews some of the prominent projects within the Fediverse.

\begin{itemize}
  \item \textbf{Mastodon}\cite{mastodon}\\
    Mastodon is a decentralized social network that operates on open-source software. It allows users to create their own servers (instances) and interact with users on other instances. Mastodon emphasizes user privacy and control over content, providing features like content warnings and robust moderation tools.

  \item \textbf{Pixelfed}\cite{pixelfed}\\
    Pixelfed is a federated image-sharing platform similar to Instagram. It focuses on user privacy and data ownership, allowing users to share photos and interact with others without centralized control. Pixelfed supports ActivityPub, enabling interaction with other Fediverse platforms.

  \item \textbf{PeerTube}\cite{peertube}\\
    PeerTube is a decentralized video hosting platform that uses peer-to-peer technology to distribute video content. It aims to provide an alternative to centralized video platforms like YouTube, giving users control over their content and reducing reliance on centralized servers. PeerTube instances can federate with each other, allowing for a distributed network of video content.

  \item \textbf{Lemmy}\cite{lemmy}\\
    Lemmy is a federated link aggregation and discussion platform similar to Reddit. It allows users to create and join communities, share links, and engage in discussions. Lemmy instances can federate with each other like it is the case with other social media in this list, enabling a decentralized network of communities that can still connect with one another.

  \item \textbf{snac}\cite{snac}\\
    SNAC is a decentralized, open-source social network that emphasizes user privacy and content control. It allows users to create their own servers (instances) and interact with others across the Fediverse. SNAC is a lightweight ActivityPub implementation with features like Mastodon API support, a simple web interface, and no database or JavaScript dependencies. Written in portable C, SNAC can be easily compiled and deployed on various platforms, providing a minimalistic alternative to mainstream social media.

\end{itemize}

\section{Decentralization and Federation}

Decentralization and federation are key concepts in the Fediverse. Decentralization refers to the distribution of data and control across multiple servers, reducing reliance on a single central authority. Federation allows different servers to communicate and interact with each other, creating a network of interconnected platforms.

\subsection{Benefits of Decentralization}

Decentralization offers several benefits, including:

\begin{itemize}
  \item \textbf{Privacy and Control}: Users have greater control over their data and privacy settings, as there is no central authority collecting and monetizing user data.
  \item \textbf{Resilience}: Decentralized networks are more resilient to censorship and outages, as there is no single point of failure.
  \item \textbf{Community Governance}: Users can create and govern their own instances, fostering diverse and self-sustaining communities.
\end{itemize}

\subsection{Challenges of Decentralization}

Despite its benefits, decentralization also presents challenges:

\begin{itemize}
  \item \textbf{Interoperability}: Ensuring seamless interaction between different platforms and instances can be complex.
  \item \textbf{Moderation}: Decentralized networks require robust moderation tools to manage content and prevent abuse.
  \item \textbf{Scalability}: Decentralized systems must be designed to handle large numbers of users and high volumes of data.
\end{itemize}

\section{Protocols and Technologies}

The Fediverse relies on various protocols and technologies to enable decentralization and federation. Some of the key protocols include:\\

\begin{itemize}
  \item \textbf{ActivityPub}\cite{ActivityPub}\\
    ActivityPub is a decentralized social networking protocol used by many Fediverse platforms. It enables users to follow, share, and interact with content across different instances.

  \item \textbf{WebFinger}\cite{webfinger}\\
    WebFinger is a protocol for discovering information about people and resources on the internet. It is used in the Fediverse to locate user profiles and instances.
\end{itemize}

\section{Conclusion}

The Fediverse represents a growing movement towards decentralized and federated social networks. By leveraging protocols like ActivityPub, platforms within the Fediverse offer users greater control over their data, enhanced privacy, and resilient communities. However, challenges such as interoperability, moderation, and scalability must be addressed to ensure the continued growth and success of decentralized social networks.

